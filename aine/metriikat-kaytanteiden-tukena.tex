\documentclass[finnish]{../tktltiki2}

% --- Packages ---

\usepackage[utf8]{inputenc}
\usepackage{lmodern}
\usepackage{microtype}
\usepackage{amsfonts,amsmath,amssymb,amsthm,booktabs,color,enumitem,graphicx}
\usepackage[pdftex,hidelinks]{hyperref}

% Automatically set the PDF metadata fields
\makeatletter
\AtBeginDocument{\hypersetup{pdftitle = {\@title}, pdfauthor = {\@author}}}
\makeatother

% --- Language ---

\usepackage[fixlanguage]{babelbib}
\selectbiblanguage{finnish}

% Add bibliography to the table of contents
\usepackage[nottoc,numbib]{tocbibind}

% tocbibind renames the bibliography, use the following to change it back
\settocbibname{Lähteet}

% --- Theorem environment definitions ---

\newtheorem{lau}{Lause}
\newtheorem{lem}[lau]{Lemma}
\newtheorem{kor}[lau]{Korollaari}

\theoremstyle{definition}
\newtheorem{maar}[lau]{Määritelmä}
\newtheorem{ong}{Ongelma}
\newtheorem{alg}[lau]{Algoritmi}
\newtheorem{esim}[lau]{Esimerkki}

\theoremstyle{remark}
\newtheorem*{huom}{Huomautus}

% --- tktltiki2 options ---

\title{Metriikat käytänteiden tukena ohjelmiston laadun\\arvioimisessa}
\author{Kasper Hirvikoski}
\date{\today}
\level{Aine - Luonnosversio}

\begin{document}

% --- Front matter ---

\maketitle

\tableofcontents
\newpage

% --- Main matter ---

\section{Johdanto}

Ohjelmistot kehittyvät elinkaarensa aikana muun muassa uusien vaatimusten, optimisaatioiden, tietoturvaparannusten ja 
virhekorjausten johdosta. Kehitysvaiheessa olevan ohjelmiston laadun varmistaminen on 
hankalaa~\cite{NB05, NB07, ZN08, MNDT09}. Ohjelmiston testaamisen ja käytännössä havaittujen virheiden välillä on usein 
suuri kuilu. Virheiden määrää ei yleensä pystytä laskemaan luotettavasti ennen kuin tuote on valmis ja julkaistu 
asiakkaalle. Tässä piilee kuitenkin ongelman ydin: virheiden korjaaminen kehityksen lopussa on usein erittäin kallista.

    On selvää, että laadun varmistaminen ja mahdollisten ongelmakohtien havaitseminen mahdollisimman aikaisessa 
vaiheessa hyödyttää kehitystyötä suuresti. Ohjelmiston koodin takana on aina ihminen,  laadun takeeksi ei voida luetella 
pelkästään mekaanisia, laatua arvioivia metriikoita. Siksi kehittäjän käytänteillä on suuri laadullinen merkitys 
ohjelmiston kaikissa kehitysvaiheissa.

\section{Laadullinen arviointi}

ISO 8402 -standardi määrittelee ohjelmiston laadun kokonaisuutena, joka kattaa tuotteen tai palvelun piirteet, jotka 
täyttävät ohjelmistoa vasten asetetut toiveet ja tarpeet. IEEE taas määrittelee laadun arviona, kuinka paljon 
ohjelmisto, järjestelmä, komponentti tai prosessi täyttää sille etukäteen määritellyt vaatimukset ja asiakkaan sekä 
käyttäjän asettamat tarpeet ja odotukset. Molemmat määritelmät painottavat vahvasti asiakkaan tarpeiden täyttämistä.

    Laadulliset kriteerit on jaettu neljään osa-alueeseen: laatumalliin, ulkoisiin, sisäisiin ja käyttölaadullisiin 
metriikoihin. Laatumalli luokittelee laadun jäsenneltynä joukkona laadullisina piirteinä ja vaatimuksina, jonka 
kehyksiin organisaatio määrittelee ohjelmistoa varten laadulliset kriteerit. Ulkoiset metriikat vastaavat ohjelmiston 
toimintaa, sisäiset metriikat pohjautuvat taas ohjelmiston sisäisiin, rakenteellisiin mittareihin. Ulkoisia metriikoita 
voidaan mitata muun muassa julkaisun jälkeisten virheiden määrällä, eli kuinka paljon virheitä tuotteessa on. Sisäisillä 
metriikoilla ohjelmiston laatua arvioidaan taas koodimetriikoilla, kuten koodin monimutkaisuudella, riippuvuuksilla ja 
muilla vastaavilla tekijöillä. Sisäinen laatu tutkii siis olennaisesti koodin laatua. Käyttölaadullisuus voidaan 
arvioida vasta kun ohjelmisto on julkaistu käyttötarkoitustaan varten.

    Laadun varmistamista rajaa ohjelmistokehityksessä aika ja raha~\cite{ZN08}. Kehittäjät kohtaavat usein tiukkoja 
määräaikoja ja rajallisia henkilöresursseja laadun takaamiseen. Johtavat käyttävät käytännössä pelkästään omakohtaisia 
kokemuksiaan resurssien tehokkaaseen jakamiseen. Yleisenä totuutena pidetään, että monimutkaisiin komponentteihin on 
syytä varata enemmän aikaa että rahaa. Tällä turvataan se, että komponenttien testaus ja tarkastus ohjataan 
haastavimpiin osa-alueisiin. Johtajilla ei kuitenkaan ole läheskään aina tarvittavaa kokemusta tai tietoa, joiden 
pohjalta he voisivat tehdä päätöksiä järkevästi. Siitä johtuen päätökset tehdään usein johtajien odotusten mukaan ja 
tällöin he itse joutuvat arvioimaan laatua puutteellisin tiedoin. Kriittiseksi osatekijäksi muodostuu näin ollen 
johtajien taito. On hyvin todennäköistä, että laadullisen arvioinnin tehokkuus ja vaatimustaso kärsivät tästä.

    Erilaiset metriikat tarjoavat yhden tehokkaan keinon ohjelmistojen laadun arviointiin. Staattisten ja dynaamisten 
virheenpaikannustekniikoiden johdosta virheiden laatu on muuttunut. Virheenpaikannustekniikoilla pyritään analysoimaan 
ohjelmiston lähdekoodia ja havaitsemaan siitä silmäänpistävimmät ''kielioppivirheet''. Tämän ansiosta suurin osa 
virhetietokantoihin tallennetuista raporteista on nykyään luonteeltaan semanttisia, eli virheet koostuvat loogisista 
ongelmista~\cite{ZN08}. Nykyään tehokkaiden metriikoiden tulee ottaa tämä huomioon.

    Nagappan ja Ball esittävät suhteellisen koodikirnu-tekniikan järjestelmän virhetiheyden ennakoimiseen~\cite{NB05}. 
Koodikirnu (code churn) mittaa ja ilmaisee määrällisesti ohjelmiston komponentteihin kohdistuvia muutoksia tietyn 
ajanjakson aikana. Nagappan ja Ball tuovat esille joukon suhteellisia koodikirnu-mittayksiköitä, jotka he rinnastavat 
muihin muuttujiin kuten komponenttien kokoon tai muokkauksen ajalliseen pituuteen. Käyttäen apuna tilastollisia 
regressiomalleja, he osoittavat suhteellisten koodikirnu-mittojen kyvyn havaita järjestelmän virhetiheyden paremmin kuin 
ehdottomien mittojen. Väittämien tueksi he suorittivat tapaustutkimuksen, jonka kohteena oli Windows Server 2003. 
Samalla he osoittavat, että relatiivinen koodikirnu pystyy paikallistamaan virheherkät komponentit 89 \% tarkkuudella.

    Monimutkaisuusmetriikoilla mitataan tyypillisesti ohjelmiston virhealttiutta. Metriikat ovat muodostettu esimerkiksi 
komponentin koodirivien, muuttujien ja metodien lukumäärästä. Niiden perimmäinen tarkoitus on arvioida kuinka 
monimutkainen jokin ohjelmiston komponentti on. Taustalla on yksinkertainen olettamus, että monimutkaisuus lisää 
ohjelmiston virheherkkyyttä~\cite{NB05, NB07, ZN08, MNDT09}. Monimutkaisuusmetriikat keskittyvät kuitenkin harvoin 
komponenttien välisiin vuorovaikutussuhteisiin.

    Ylläpidettävyyden vuoksi ohjelmiston lähdekoodia on hyvä eriyttää mahdollisimman paljon niiden tarjoamien 
palveluiden mukaan. Ohjelmiston komponentit riippuvat lähes aina toisista komponenteista, jolloin ne käyttävät niiden 
tarjoamia palveluita tuottaakseen oman toiminnallisuutensa. Järjestelmän riippuvuudet voidaankin esittää matalan tason 
verkkoina, jossa komponenttien keskinäiset suhteet paljastuvat. Niistä ilmenee mitä osia esimerkiksi komponentti A 
tarvitsee toiminnalleen ja myös toisinpäin, mitkä osat tarvitsevat A:n palveluita.

    Zimmermann ja Nagappan esittävät verkkoanalyysin suorittamista komponenttien riippuvuusverkoille~\cite{ZN08}. 
Verkkoanalyysillä voidaan paikallistaa ohjelmiston kriittiset komponentit, jotka ovat muita virheherkempiä. Tämä 
tapahtuu tutkimalla verkkoa sekä kokonaisuutena että osina (aliverkot) erilaisten verkkometriikoiden pohjalta. 
Ohjelmistojen kohdalla komponentit muodostavat verkon toimijat ja komponenttien väliset riippuvuudet niiden väliset 
vuorovaikutukset.

    Zimmermann ja Nagappan vertasivat verkkoanalyysia ja monimutkaisuusmetriikoita Windows Server 2003:n laadun 
arvioimisessa. Verkkoanalyysillä saavutetaan heidän tapaustutkimuksensa tilastollisten analyysien mukaan 10 \% parempi 
hyötyaste kuin pelkillä komponenttien monimutkaisuutta mittaavilla metriikoilla. Zimmermann ja Nagappan havaitsivat 
myös, että verkkometriikat pystyvät identifioimaan 60 \% komponenteista, joita ohjelmiston kehittäjät pitivät 
kriittisinä ohjelmiston kannalta. Tämä on jopa kaksi kertaa enemmän kuin tavallisilla monimutkaisuusmetriikoilla 
saavutettavat tulokset. 

    Ohjelmiston kehityksessä koodin testaaminen on kriittinen osa laadun takaamista. Testaamisessa ohjelmiston 
lähdekoodi alistetaan testitapauksille, joiden tarkoitus on kattaa ja varmistaa mahdollisimman hyvin kaikki loogiset 
tilanteet, jotka ohjelmisto käy tietyssä tilanteessa läpi. Parhaassa tapauksessa testit löytävät virheet ohjelmistosta 
ennen kuin se julkaistaan asiakkaalle.

    Mockus, Nagappan ja Dinh-Trong tutkivat testien laadullista arviointia keinona havaita virheherkkiä komponentteja 
ohjelmistosta~\cite{MNDT09}. Testien analysoimisessa tulisi keskittyä nimenomaan niiden kykyyn havaita mahdollisia 
virheitä ohjelmistosta. On selvää, että taitavat kehittäjät todennäköisesti tuottavat laadukkaampia testejä, mutta 
testien arvioiminen määrällisesti ja laadullisesti on kannattavampaa toteuttaa automaattisesti. Yleisin testien 
tehokkuutta arvioiva mittari on testikattavuus.

    Testikattavuuden lajeja on useita. Yksinkertaisista luokka-, funktio/metodi- ja käskykattavuuksista kehittyneisiin 
haara- ja polkukattavuuksiin. Nimensä mukaan kukin laji testaa lähdekoodin eri osa-alueita. Funktio-/metodikattavuu\-della 
kartoitetaan testien kattavuutta yhden toiminnallisuuden osalta, luokkakattavuudella taas näistä muodostuvan 
kokonaisuuden testien kattavuutta. Taustalla on olettamus, että jos jokin yksittäinen looginen ehto tai polku ei ole 
katettu vähintään yhdellä testillä, ei sen mahdollisesti sisältämiä virheitä pystytä havaitsemaan~\cite{MNDT09}.

    Voidaankin olettaa, että suurempi testikattavuus löytää todennäköisesti enemmän virheitä ja takaa näin ollen 
paremman laadun. Kattavuus kuvaa yksinkertaisesti osuutta siitä kuinka monta riviä ohjelmakoodia on katettu sitä 
testaavalla testikoodilla. Kattavuudella ei kuitenkaan pystytä arvioimaan kuinka todennäköisesti nämä rivit aiheuttavat 
virheen, siksi suuri testikattavuus ei yksinään takaa laatua. On kuitenkin selvää, että testikattavuus auttaa laadun 
takaamisessa. Muita metriikoita tulisi käyttää kohdentamaan testejä ohjelmiston kriittisiin 
osa-alueisiin~\cite{NB07, MNDT09}.

    Ohjelmiston koodin takana on lopulta aina ihminen.  Laadun takeeksi ei voida luetella pelkästään mekaanisia laatua 
arvioivia metriikoita. Kehittäjän käytänteillä on suuri laadullinen merkitys ohjelman kaikissa kehitysvaiheissa, joten 
ohjelmistotuotantomenetelmät nousevat suureen rooliin. Niiden tulisi ohjata laadukasta kehitystä.

    Vanhojen raskaaseen ennakkosuunniteluun pohjautuvien tuotantomenetelmien, kuten vesiputousmallin, rinnalla on 
noussut uusia ketterän kehityksen malleja. Ne painottavat yksilöitä ja yksilöiden vuorovaikutusta, toimivan ohjelmiston 
merkitystä, asiakkaan merkitystä kehitysprosessin kriittisenä osana ja muutoksiin sopeutuvaa kehitystä~\cite{BBB01}. 
Ketterässä kehityksessä ohjelmisto tuotetaan iteratiivisesti, pala kerrallaan, sopeutuen uusiin tavoitteisiin. 
Vesiputousmallissa, jossa ohjelmisto suunnitellaan tiukasti ennen toteutusta, lopputuotokset eivät yleensä vastaa 
haluttua tulosta, varsinkaan asiakkaan kannalta. Useat empiiriset tutkimukset tukevat ketterien kehitysmallien hyötyä 
merkittävänä laadullisina tekijänä~\cite{SS10}.

\section{Metriikat}

\subsection{Koodikirnu}

Nagappan ja Ball esittävät ohjelmistojen virhetiheyden arvioimiseen ratkaisuksi koodikirnua~\cite{NB05}. Se mittaa 
ohjelmiston komponenttien ohjelmakoodiin kohdistuvien muutosten määrää tietyn ajanjakson aikana. Muutosten määrä on 
helposti saatavilla ohjelmiston versionhallintajärjestelmien muutoshistoriasta. Useimmat versionhallintajärjestelmät 
vertailevat lähdekooditiedos\-tojen historiaa ja laskevat automaattisesti koodiin kohdistuvia muutoksia. Nämä muutokset 
ilmentävät kuinka monta riviä tiedostoon on ohjelmoijan toimesta lisätty, poistettu tai muutettu sitten viimeiseen 
versiohistoriaan tallennetun version. Nämä muutokset muodostavat koodikirnun pohjan.

    Nagappan ja Ball esittävät joukon suhteellisia koodikirnu-mittoja virhetiheyden havaitsemiseen. Mitat ovat 
normalisoituja arvoja koodikirnun aikana saaduista tuloksista. Normalisoinnilla niistä on pyritty poistamaan mahdolliset 
häiriötekijät. Näitä mittoja on muun muassa yhteenlaskettujen koodirivien määrä, tiedostojen muutokset ja tiedostojen 
määrä. Tutkimukset ovat osoittaneet, että ehdottomat mittayksiköt, kuten pelkkä koodirivien summa, ovat huonoja ohjelmiston 
laadullisia ennusteita. Yleisesti ottaen ohjelmiston kehitysprosessia mittaavien yksiköiden on havaittu olevan parempia
osoittimia vikojen määrästä kuin pelkkää koodia arvioivat tekijät.

    Ohjelmistoa kehitettäessä sen komponenttien monimutkaisuus muuttuu. Monimutkaisuuden kasvun suhde on hyvä mittari 
virheherkkyyden kasvulle. Koodikirnu-mittojen on havaittu korreloivan selvästi ohjelmistoista tehtyjen vikailmoitusten 
kanssa. Mittojen välillä on havaittavissa myös keskinäisiä suhteita, joita voidaan mallintaa verkkoina. Yksinään 
kyseiset mittarit eivät välttämättä tuota toivottua tulosta. Näin ollen mittoja verrataan keskenään mahdollisten 
ristiriitaisuuksien havaitsemiseksi. On kuitenkin selvää, että johtopäätöksiin päätyminen on hankalaa empiirissä 
tutkimuksissa, koska prosessien taustalla on usein laajoja kontekstisidonnaisia tekijöitä.

\subsubsection{Ohjelmiston virheherkkyyteen vaikuttavat mitat}

Nagappan ja Ball listaavat seuraavat ehdottomat mitat koodikirnun pohjaksi. Nämä muodostavat suhteellisille mitoille 
vertailukohdat ohjelmiston virheherkkyyden analysoimisessa. Ehdottomat mitat eivät yksinään tuota luotettavaa tulosta.

\begin{description}
    
    \item[Yhteenlaskettu koodirivien määrä,] ohjelman uuden version ei-kommen\-toitujen koodirivien summa kaikkien 
                                             lähdekooditiedostojen kesken.
    
    \item[Käsiteltyjen koodirivien määrä,] ohjelman lähdekoodiin lisättyjen ja muuttuneiden koodirivien summa 
                                           edelliseen versioon nähden.
    
    \item[Poistettujen koodirivien määrä,] ohjelman lähdekoodista poistettujen koodirivien määrä edelliseen versioon 
                                           nähden.
    
    \item[Tiedostojen määrä,] yhden ohjelman kääntämiseen tarvittavien lähdekoodi\-tiedostojen määrä.
    
    \item[Muutosten ajanjakso,] yhteen tiedostoon kohdistuneiden muutosten ajanjakson pituus.
    
    \item[Muutosten määrä,] ohjelman tiedostoihin kohdistuneiden muutosten määrä edelliseen versioon nähden.
    
    \item[Käsiteltyjen tiedostojen määrä,] ohjelman käsiteltyjen tiedostojen yhteenlaskettu määrä.

\end{description}

    Näiden pohjalta he muodostivat kahdeksan suhteellista koodikirnu-mittaa ja osoittivat, että nämä mitat korreloivat 
selvästi kohonneeseen virhemäärään koodirivejä kohden. He käyttivät Spearmanin järjestyskorrelaatiokerrointa, joka kuvaa 
kahden asian keskinäistä vastaavuutta. Analyysissään he havaitsivat suhteellisten mittojen ylivertaisuuden ehdottomiin 
verrattuna. Empiiristen tutkimusten avulla he havaitsivat seuraavien mittojen soveltuvuuden todellisen virhetiheyden 
ennakoimiseen.

\begin{enumerate}
    
    \item {\bf Käsiteltyjen koodirivien määrä / Yhteenlaskettu koodirivien määrä}
    
    Suurempi osa käsiteltyjä koodirivejä suhteessa yhteenlaskettuun koodirivien määrän vaikuttaa yksittäisen ohjelman 
    virhetiheyteen.
    
    \item {\bf Poistettujen koodirivien määrä / Yhteenlaskettu koodirivien määrä}
    
    Suurempi osa poistettuja koodirivejä suhteessa yhteenlaskettuun koodirivien määrään vaikuttaa yksittäisen ohjelman 
    virhetiheyteen. Nagappan ja Ball havaitsivat korkean korrelaation mittojen 1. ja 2. välillä.
    
    \item {\bf Käsiteltyjen tiedostojen määrä / Tiedostojen määrä}
    
    Suurempi osa käsiteltyjä tiedostoja suhteessa ohjelman rakentavien tiedostojen lukumäärään lisää todennäköisyyttä, 
    että nämä käsitellyt tiedostot aiheuttavat uusia vikoja. Esimerkiksi meillä on kaksi ohjelmaa A ja B, jotka molemmat 
    koostuvat 20 lähdekooditiedostosta. A sisältää viisi käsiteltyä tiedostoa ja B kaksi. Todennäköisyys sille, että 
    muutokset ohjelmaan A saattavat aiheuttaa uusia vikoja on siis suurempi.
    
    \item {\bf Muutosten määrä / Käsiteltyjen tiedostojen määrä}
    
    Mitä suurempi määrä yksittäisiin tiedostoihin on kohdistunut muutoksia, sitä suurempi on todennäköisyys sille, että 
    tämä vaikuttaa kyseisistä lähdekooditiedostoista muodostuvan ohjelman virhetiheyteen. Esimerkiksi jos ohjelman A 
    viittä lähdekooditiedostoa on muutettu 20 kertaa ja ohjelman B viittä tiedostoja on muutettu kymmenen kertaa, 
    todennäköisyys sille, että ohjelma A sisältää uusia vikoja on suurempi.

    \item {\bf Muutosten ajanjakso / Tiedostojen määrä}
    
    Mitä pitempi aika on kulutettu muutoksiin, jotka kohdistuvat pieneen joukkoon tiedostoja, sitä suurempi on 
    todennäköisyys sillä, että nämä tiedostot sisältävät monimutkaisia rakenteita. Monimutkaisuus vaikuttaa koodin 
    helppoon ylläpidettävyyteen ja näin ollen lisää näiden tiedostojen aiheuttamaa virhetiheyttä.

    \item {\bf Käsiteltyjen ja poistettujen koodirivien määrä / Muutosten ajanjakso}
    
    Käsiteltyjen ja poistettujen koodirivien määrä suhteessa muutosten ajanjaksoon mittaa muutoksen määrää, jota pelkkä 
    muutosten ajanjakso ei yksinään ilmaise. Tätä mittaa tulee verrata mittaan 5. Oletuksena on, että mitä suurempi 
    määrä käsiteltyjä ja poistettuja koodirivejä on, sitä pitempi muutosten ajanjakson tulisi olla. Tämä taas vaikuttaa 
    ohjelman virhetiheyteen.

    \item {\bf Käsiteltyjen koodirivien määrä / Poistettujen koodirivien\\määrä}
    
    Ohjelmiston kehitys ei koostu pelkästään vikojen korjaamisesta vaan myös uuden kehittämisestä. Uusien ominaisuuksien 
    kehittämisessä käsiteltyjen koodirivien määrä on suhteessa suurempi kuin poistettujen koodirivien määrä. Suuri arvo 
    tälle mitalle ilmaisee uutta kehitystä. Saatua arvoa verrataan mittoihin 1. ja 2., jotka yksinään eivät ennakoi 
    uutta kehitystä.

    \newpage

    \item {\bf Käsiteltyjen ja poistettujen koodirivien määrä / Muutosten määrä}
    
    Mitä suurempi muutoksen laajuus on suhteessa muutosten määrään, sitä suurempi virhetiheys on. Mitta 8. toimii
    verrokkina mitoille 3. -- 6. Suhteessa mittoihin 3. ja 4., mitta 8. ilmaisee todellisen muutoksen määrää. Se kompensoi
    sitä tietoa, että yksittäisiä tiedostoja ei käsitellä toistuvasti pienten korjausten takia. Suhteessa mittoihin 5. 
    ja 6.,mitä suurempi käsiteltyjen ja poistettujen koodirivien määrä on käsittelyä kohden, sitä pitempi muutosten 
    ajanjakso tarvitaan ja sitä enemmän muutoksia kohdistuu esimerkiksi jokaista viikkoa kohden. Muussa tapauksessa 
    suuri määrä muutoksia on saattanut kohdistua hyvin lyhyeen ajanjaksoon, joka ennakoi suurempaa virhetiheyttä.

\end{enumerate}

\subsubsection{Johtopäätökset koodikirnusta}

Nagappan ja Ball havaitsivat, että koodi joka muuttuu useasti ennen julkaisua on selvästi virheherkempää kuin koodi, 
joka muuttuu vähemmän saman ajanjakson aikana. He tutkivat kahden ohjelmistojulkaisun Windows Server 2003 ja Windows 
Server 2003 Service Pack 1 pohjalta saatuja tuloksia. Julkaisuista analysoitiin 44,97 miljoonaa riviä koodia, joka 
muodostui 96 189 lähdekooditiedostosta. Niistä käännettiin 2 465 yksittäistä ohjelmaa.\newline

\noindent He päätyivät tutkimuksessaan neljään johtopäätökseen:

\begin{enumerate}

    \item Suhteellisten koodikirnu-mittojen nousua seuraa ohjelmiston virheherkkyyden kasvu.

    \item Suhteelliset mitat ovat parempia laadullisia arvioijia kuin ehdottomat mitat.

    \item Suhteellinen koodikirnu on tehokas tapa arvioida ohjelmiston virheherkkyyttä.

    \item Suhteellinen koodikirnu pystyy havaitsemaan virheherkän ja toimivan komponentin toisistaan.

\end{enumerate}

\subsubsection{Koodikirnun pätevyyteen vaikuttavia tekijöitä}

Nagappan ja Ball toteavat, että mittausvirheet vaikuttavat arvion luomiseen. Ongelma ei ole kuitenkaan suuri, sillä 
versionhallintajärjestelmät hoitavat automaattisesti analyysiin vaadittavat lähtöarvot. Koodikirnu vaatii kuitenkin myös 
ohjelmiston kehittäjältä hyviä käytäntöjä. Jos kehittäjä on tehnyt useita muutoksia rekisteröimättä niitä 
versionhallintajärjestelmän historiaan, osa muutoksista jää näkemättä. Kehittäjän toimista riippuen myös muutosten 
ajanjakson pituus voi selvästi pidentyä, jos muutoksia ei hyväksytä tarpeeksi aikaisin versionhallintajärjestelmään. 
Mittojen vertaaminen keskenään lieventää tästä johtuvia poikkeamia.

    He toteavat myös, että tapaustutkimuksen pätevyyteen voidaan nähdä vaikuttavan myös sen tosiasian, että 
tutkimuksessa analysoitiin vain yhtä ohjelmistojärjestelmää. Tämä ohjelmistojärjestelmä kuitenkin koostuu lukuisista 
komponenteista ja suuresta määrästä koodia. Analyysi on siis itsessään erittäin kattava.

\subsection{Verkkoanalyysi}

Ohjelmiston komponentit riippuvat lähes aina toisista komponenteista. Järjes\-telmän riippuvuudet voidaankin esittää 
matalan tason verkkoina, jossa komponenttien keskinäiset suhteet paljastuvat. Zimmermann ja Nagappan esittävät 
verkkoanalyysin suorittamista riippuvuusverkoille ohjelmiston virheherkkyyden arvioimiseksi~\cite{ZN08}. 
Verkkoanalyysillä voidaan paikallistaa ohjelmiston kriittiset komponentit, jotka ovat oletettavasti muita 
virheherkempiä.

    Zimmermann ja Nagappan tutkivat julkaisun jälkeisten virheilmoitusten ja riippuvuusverkkojen suhdetta ja havaitsivat 
kaksi olennaista tekijää. Selvästi keskeisessä roolissa olevat komponentit sekä yksittäiset komponentit, joilla on suuri 
määrä keskinäisiä riippuvuuksia, ovat yleisesti herkempiä virheille.

    Riippuvuus ohjelmistoissa on suunnattu yhteys kahden koodiosan kuten lausekkeen tai metodin välillä. Riippuvuudet 
voidaan erotella toisistaan: tietoriippuvuus on määrittelyiden ja arvojen välinen yhteys ja kutsuriippuvuus on 
funktio-/metodimäärittelyiden ja niitä kutsuvien paikkojen välinen yhteys.

    Zimmermann ja Nagappan löysivät useita piirteitä verkoista, joilla keskeisessä roolissa olevat komponentit voidaan 
havaita samankaltaisista aliverkoista. Yksi on niin kutsuttu tähtipiirre (star pattern), joka on komponenteilla joilla 
on useita satelliittikomponentteja. Satelliitit ympäröivät tähteään ja riippuvat yksinään siitä. Suurimmassa osassa 
tämän piirteen omaavista aliverkoista tähtikomponentti oli virheherkkä kun taas satelliitit eivät. Verkkoanalyysin 
mukaan tähtikomponentti on keskeinen komponentti, jos se ohjaa satelliittejaan. Tämänkaltaista tähtikomponenttia 
kutsutaan usein välittäjäksi.

    Mitä suurempi joukko verkossa olevia komponentteja riippuu keskenään toisistaan (clique), sitä suurempi on näiden 
komponenttien todennäköinen virheherkkyys. Riippuvuuden suunnalla ei tässä tapauksessa ole väliä. X komponentti voi 
riippua Y:stä, Y komponentti X:stä tai molemmat yhtäaikaisesti toisistaan. Joukkoa kutsutaan maksimaaliseksi, jos yhtään 
komponenttia ei voida lisätä tähän aliverkkoon siten, että maksimaalisuus ei säilyisi.

    Zimmermann ja Nagappan kävivät läpi maksimaaliset joukot Windows Server 2003:n riippuvuusverkosta ja järjestivät 
nämä joukon koon mukaan. Virheilmoituksia vertaamalla he havaitsivat, että mitä enemmän riippuvuuksia joukossa on, sitä 
enemmän virheilmoituksia korreloi näihin komponentteihin. He toteavat yhdeksi syyksi sen, että nämä joukot ovat tämän 
tiedon valossa muita monimutkaisempia riippuvuuksiensa johdosta.

    Aikaisempien tutkimusten mukaan on havaittu, että koodikirnu ja riippuvuusverkot ovat yhdessä hyviä metriikoita 
arvioida ohjelman virheherkkyyttä~\cite{NB07}. Jos komponentti B muuttuu paljon eri versioiden välillä, voidaan olettaa, 
että komponentin A täytyy muuttua myös, jotta muutokset B:hen ovat mahdollisia. Muutos yleensä leviää riippuvuuksien 
välillä.

    Riippuvuusverkossa näkyvät yhteydet ilmentävät kuinka paljon työtä tarvitaan yhteyksien ylläpitämiseen. Muutosten 
määrän lisäksi yhteydet voivat kertoa virheherkkyydestä muutakin tärkeää tietoa. Lähdekoodi ei muodosta pelkästään 
yksittäisistä komponenteista vaan arkkitehtuurista josta koko ohjelmisto koostuu. Näitä asioita arvioimalla pystytään 
paikallistamaan muita virheherkempiä komponentteja ja kohdistamaan fokus testejä ja koodikatselmusta varten.

\subsubsection{Ohjelmiston virheherkkyyteen vaikuttavat verkkomitat}

    Zimmermann ja Nagappan suorittivat verkkoanalyysin Windows Server 2003:n riippuvuusverkolle. He keräsivät samalla 
lukuisia monimutkaisuusmetriikoita ja vertasivat niitä verkkoanalyysiin. Tapaustutkimuksessaan he keskittyivät 
pelkästään riippuvuuksien esiintymiin. Kahden komponentin keskinäisten riippuvuuksien määrä jätettiin huomioimatta, 
toisin sanoen tutkimuksessa ei huomioitu jos esimerkiksi A komponentti riippuu kolme kertaa B:stä. Spearmanin ja 
Pearsonin järjestyskorrelaatiokertoimien avulla Zimmermann ja Nagappan havaitsivat seuraavien verkkomittojen korreloivan 
merkittävästi tai positiivisesti julkaisun jälkeen ilmoitettuihin virheisiin.

\begin{enumerate}
    
    \item {\bf Egoverkot}
    
    Jokaisella solmulla, eli komponentilla, on verkossa sitä vastaava egoverkko, joka kuvaa miten kyseinen solmu on 
    kytketty naapurisolmuihinsa. Komponentteja, jotka riippuvat solmusta kutsutaan sisäsolmuiksi, ja komponentteja 
    joista solmu itse riippuu kutsutaan ulkosolmuiksi. Egoverkko on sisä- ja ulkosolmujen muodostama aliverkko. Se 
    mahdollistaa komponentin paikallisen tärkeyden mittaamista suhteessa naapureihinsa.
    
    Egoverkolle voidaan suorittaa useita mittauksia. Tehokkaimpia metriikoita olivat tutkimuksen mukaan mittarit, jotka 
    kohdistuvat muun muassa egoverkon kokoon, komponenttien siteisiin, pareihin ja tiheyteen sekä verkon, eli 
    riippuvuuksien, polkuihin. Zimmermann ja Nagappan havaitsivat tutkimuksessaan, että ulkosolmut ovat sisäsolmuja 
    virheherkempiä. Toisin sanoen komponentit, joista muut komponentit riippuvat, ovat virheherkempiä.
    
    \item {\bf Globaaliverkot}
    
    Globaaliverkko muodostuu koko ohjelmiston riippuvuusverkosta ja sen muodostavien komponenttien välisistä 
    riippuvuuksista. Globaaliverkosta voidaan tutkia yksittäisen solmun tärkeyttä koko ohjelmiston kannalta ja näin 
    havaita koko ohjelmiston kriittisimmät komponentit. Keskeisessä roolissa olevat komponentit ovat tutkimuksen mukaan 
    muita virheherkempiä.
    
    \item {\bf Rakenteelliset puutteet}
    
    Ideaalisesti solmujen väliset riippuvuudet ovat toisiinsa nähden tasapainossa. Mikäli meillä on solmut A, B ja C ja 
    kaikilla on keskinäinen suhde toisiinsa, vallitsee solmujen välillä tasapaino. Jos kuitenkin esimerkiksi solmujen B 
    ja C välillä ei ole keskinäistä riippuvuutta, vaan ne riippuvat toisistaan A:n kautta, solmulla A on selvä 
    etulyöntiasema. A toimii välittäjänä B:n ja C:n välillä ja näin ollen solmut eivät ole tasapainossa. Välittäjäsolmut 
    ovat Zimmermannin ja Nagappanin tutkimuksen mukaan muita virheherkempiä.
    
    \item {\bf Keskeisyys}
    
    Yleisin verkkomitoista on solmun keskeisyys. Sillä pyritään havaitsemaan komponentit jotka ovat suotuisassa 
    asemassa, eli useat muut komponentit riippuvat kyseisestä komponentista. Keskeisyyttä voidaan mitata riippuvuuksien 
    määrällä, komponenttien riippuvuuksien etäisyyksillä toisistaan ja komponentista johtavien riippuvuuspolkujen 
    piirteillä. Mikäli komponentti riippuu hyvin suuresta määrästä toisia komponentteja, on se todennäköisesti muita 
    komponentteja virheherkempi.
    
    Komponentit, joiden riippuvuuksien välillä on hyvin lyhyet polut, voidaan todeta olevan myös muita virhealttiimpia. 
    Muutokset näihin komponentteihin yleensä leviävät myös komponenttien riippuvuuksiin.
    
    Keskeisyydellä on tutkimuksen mukaan myös vastakkainen vaikutus. Keskeisyys saattaa tehdä komponenteista jopa 
    vähemmän virheherkkiä. Oletettavasti näihin komponentteihin on kehitystyössä luultavasti panostettu enemmän, joka 
    parantaa niiden laatua.
    
\end{enumerate}

\subsubsection{Johtopäätökset verkkoanalyysistä}

Windows Server 2003:n kehittäjätiimit pitivät listaa ohjelmistojärjestelmän kriittisistä komponenteista. Kehittäjät 
valitsevat käsin nämä komponentit niihin liittyvän historian perusteella. Arvioon vaikuttuvat muun muassa 
komponentteihin kohdistuneet muutokset ja virheiden määrä. Kriittisiä komponentteja tulee varjella muita tarkemmin. 
Jos esimerkiksi Windowsin ytimeen tehdään muutoksia, täytyy muutokset varmentaa kattavien testausten ja tarkastelujen 
avulla.

    Zimmermann ja Nagappan vertaisivat verkkoanalyysin havaitsemia komponentteja kehittäjien laatimaan kriittisten 
komponenttien listaan. Monimutkaisuusmetriikat löysivät kriittistä komponenteista vain 30 \%, kun verkkoanalyysillä 
saavutettiin kaksi kertaa parempi tulos.

    Yksittäisissä tapauksissa monimutkaisuusmetriikat olivat toisaalta jopa hieman parempia kuin verkkometriikat. 
Olio-ohjelmointiin liittyvät monimutkaisuusmetriikat eivät kuitenkaan sovellu epäyhtenäisten ohjelmistojen arvioimiseen. 
Ohjelmiston täytyy noudattaa täysin olio-paradigmaa, jotta metriikoilla olisi merkitystä.

    Zimmermann ja Nagappan pitävät verkkoanalyysin suorittamista ohjelmiston komponenttien riippuvuusverkoille 
tehokkaana tapana arvioida ohjelman virheherkkyyttä.\newline

\noindent He päätyivät tutkimuksessaan kolmeen johtopäätökseen:

\begin{enumerate}

    \item Verkkometriikat riippuvuusverkoissa pystyvät löytämään kriittisiä komponentteja, joita pelkät 
          monimutkaisuusmetriikat eivät havaitse.
          
    \item Verkkometriikoiden antamat arvot riippuvuusverkoissa korreloivat julkaisun jälkeen ilmoitettujen virheiden 
          kanssa. Suurempaa arvoa johtaa todennäköisesti suurempi virhetiheys.
          
    \item Verkkometriikat riippuverkoissa pystyvät arvioimaan julkaisun jälkeisten virheiden määrää.

\end{enumerate}

\subsubsection{Verkkoanalyysin pätevyyteen vaikuttavia tekijöitä}

Zimmermann ja Nagappan olettivat tutkimuksessaan, että virheet ja niiden korjaukset sijoittuvat lähdekoodissa samaan 
paikkaan. He kuitenkin toteavat, että näin ei aina ole, mutta tämä olettamus on yleisesti käytössä tutkimuksissa.

    Yleisesti pätevien päätelmien tekeminen empiirisistä tutkimuksista on vaikeaa niiden kontekstisidonnaisen luonteen 
takia. Virheherkkyyden arvioimiseen ei ole löytynyt yksittäistä ''parasta'' ratkaisua, joten verkkoanalyysin tuottamia 
tuloksia ei pystytty vertailemaan sellaisen tehokkuuteen. Tulokset antavat kuitenkin lupaavia viitteitä.

    Zimmermann ja Nagappan toteavat myös, että tapaustutkimukseen voi vaikuttaa se tosiasia, että tutkimuksessa 
analysoitiin vain yhtä ohjelmistojärjestelmää. Tämä järjestelmä on kuitenkin kooltaan suurempi kuin useat muut 
kaupalliset ohjelmistot. Tästä johtuen verkkoanalyysin pitäisi soveltua virheherkkyyden arvioimiseen myös muissa 
ohjelmistoissa. 

    Verkkoanalyysi ei todennäköisesti sovellu yksinään virheherkkyyden arvioimiseen. Sen voidaan kuitenkin nähdä olevan 
osa palapeliä. Tutkimuksessa esille tulleet monimutkaisuusmetriikat ja koodikirnu ovat Zimmermannin ja Nagappanin mukaan 
varteenotettavia lisiä verkkoanalyysin tehokkuudelle. Mikään kyseisistä metriikoista ei kuitenkaan ota huomioon 
virheiden inhimillistä tekijää. Kehittäjät loppujen lopuksi aiheuttavat virheet itse kehitystyön tuloksena.

\subsection{Testikattavuus}

Mockus, Nagappan ja Dinh-Trong tutkivat testien laadullista arviointia keinona havaita virheherkkiä 
komponentteja~\cite{MNDT09}. Testien arvioimisessa tulisi nimenomaan keskittyä testien kykyyn havaita mahdollisia 
virheitä ohjelmistosta. Parempien testien tulisi löytää enemmän mahdollisia ongelmakohtia ohjelmiston lähdekoodista ja 
näin johtaa ohjelmiston parempaa laatuun.

    Mockus ym. tekivät tapaustutkimuksen kahden eri organisaation hyvin erilaisista ohjelmistosta: Microsoft Windows 
Vista:sta ja Avaya:sta. Ensimmäinen ohjelmisto on toteutettu C ja C++ kielillä, toinen Javalla. Windows Vistan 
testaamiseen käytettiin Microsoftin sisällä kehitettyä työkalua ja Avayan testaamiseen JUnit-testiympäristöä. 
Testikattavuus keskittyi eri ohjelmointikielistä johtuen näiden kielten eri piirteisiin.

    Mockus ym. havaitsivat tutkimuksessaan, että molempien ohjelmistojen osalta suurempaa testikattavuutta seurasi 
pienempi määrä julkaisun jälkeisiä virheilmoituksia. He huomasivat myös, että suuremman testikattavuuden saavuttaminen 
kasvaa eksponentiaalisesti, mitä suurempiin testikattavuuksiin tähdätään. Samalla virheherkkyys vähenee vain 
lineaarisesti. 

    Optimaalinen testikattavuus ei tunnu olevan lähelläkään 100 \%, eikä sen saavuttaminen ole ohjelmiston laadun 
kannalta välttämätöntä, saati tehokasta. He käyttivät tutkimuksessaan Spearmanin järjestyskorrelaatiokerrointa 
testikattavuuden ja julkaisun jälkeisten virheiden korrelaation selvittämiseen.

    Testikattavuuden kannalta olisi mielenkiintoista tietää kuinka suuri osa itse testeistä havaitsi virheet. Suurin osa 
tästä työstä tapahtuu kuitenkin kehittäjän toimesta kehitysvaiheessa yksikkötestauksen tasolla. Viitteet tästä eivät 
tallennu versionhallintaan eikä virheitä ilmoiteta virhetietokantoihin. Tämän korrelaation tutkiminen on valitettavan 
haasteellista.

    Mockus ym. havaitsivat tutkimuksessaan, että lähdekoodin monimutkaisuus, ohjelmiston käyttökohde, kehittäjien taso 
ja etätyöskentely vaikuttavat myös ohjelmiston virheherkkyyteen sekä testien kattavuuteen. Vähemmän kokeneilla 
kehittäjillä ja etätyöskentelyssä testikattavuuden merkitys kasvaa selvästi. 

\subsubsection{Testikattavuuden vaikutus ohjelmiston virheherkkyyteen}

Mockus ym. tutkimus analysoi versionhallintajärjestelmistä, virhetietokannoista ja testikattavuudesta saatavia tietoja 
arvioidakseen kunkin osa-alueen vaikutusta ohjelmiston virheherkkyyteen. Samalla ohjelmistoista laskettiin 
monimutkaisuusmetriikoita ja tutkittiin lähdekoodin muutosten määrää. Aikaisempien tutkimusten mukaan muutosten määrä on 
varteenotettava mittari virheherkkyyden arvioimiseen, siksi testikattavuus on syytä suhteuttaa siihen. Tilastollisten 
analyysien lisäksi he myös haastattelivat kehittäjiä tulosten vahvistamiseksi.

    Avaya -tapauksessa Mockus ym. havaitsivat selvän korrelaation testikattavuuden kasvulla ja julkaisun jälkeisten 
virheilmoitusten vähenemisellä. Lähdekooditiedostoja joita testit eivät kattaneet, löytyi selvästi eniten 
virheilmoituksia. Vastaavasti tiedostoille, joiden testikattavuus oli vähintään 50 \%, virheiden määrä oli selvästi 
pienempi. Testikattavuuden teho kuitenkin ryhtyy vähenemään jo 60 \% kattavuuden kohdalla.

    Versionhallintajärjestelmistä pystytään analysoimaan testikattavuuden saavuttamiseen käytetty aika, eli kuinka kauan 
kehittäjän on täytynyt käyttää yksittäisen komponentin automaattiseen testaamiseen. Kattavimpiin testeihin kuluu 
eksponentiaalisesti pidempi aika mitä korkeampia testikattavuuksia tavoitellaan. Tämä indikoi, että täyden 
testikattavuuden tavoittaminen ei kaikissa tapauksista ole välttämättä hyödyllistä. Tämä voi johtua myös siitä, että 
tiedostot jotka muuttuvat eniten ovat testattu kattavammin, sillä muutokset helposti tuovat uusia virheitä. 50 \% 
testikattavuuden saavuttaminen näyttää tulosten pohjalta olevan suhteellisen helppoa. Mockus ym. vertasivat Avayasta 
saatuja tuloksia ja havaitsivat, että samat johtopäätökset voitiin tehdä myös Windows Vistan tapauksessa.

    Tilastollisen analysoinnin tulosten tueksi Mockus ym. suorittivat haastattelun tuotteiden kehittäjätiimeissä. 
Kehittäjät, jotka kirjoittavat lähdekoodin alusta asti, testaavat yleensä koodin kattavammin. Kehittäjät, jotka vain 
ylläpitävät toisten kirjoittamaa koodia, harvoin kasvattavat koodin testikattavuutta. Selvästi loogisesti monimutkaiset 
ja helpot lähdekooditiedostot testataan usein muita kattavammin. Myös komponentit, jotka tarjoavat palveluita useille 
muille komponenteille testataan haastattelun pohjalta kattavammin.

    Keskeisessä roolissa olevista komponenteista saatetaan toisaalta löytää enemmän virheitä puhtaasti sen takia, että 
niitä käytetään useammin. Nämä komponentit joutuvat tiukempiin käytännöntilanteisiin liittyviin testauksiin. 
Testikattavuuden saavuttaminen on selvästi heikompaa käyttöliittymään ja tietokantaan liittyvissä koodiosuuksissa, koska 
näiden testaamista pidetään kehittäjien keskuudessa muita haastavampana. Haastattelussa havaittiin myös, että 
etätyöskentely tuntuu vähentävän testauksen määrää.

\subsubsection{Johtopäätökset testikattavuudesta}

Testikattavuuden lähtökohtana on se, että lähdekoodin virheitä ei pystytä havaitsemaan ellei kyseisiä rivejä testata 
vähintään yhdellä testillä. Testikattavuuden ja laadun välistä yhteyttä on tutkittu yllättävän vähän, varsinkaan 
laadullisista näkökulmista. Testikattavuutta tulisi ohjata komponenttien tärkeysjärjestyksen pohjalta. Selvästi 
kriittisempiä osia tulisi painottaa testeissä, unohtamatta siltikään pienemmissä osissa olevia osia.

    Ongelmana on kuitenkin se, että vakavimmat virheet voidaan havaita jo hyvin pienellä testikattavuudella. Vaikka 
jokin yksittäinen rivi lähdekoodista olisi katettu testeillä, ei se takaa sitä, että tämä testi pystyisi havaitsemaan 
kyseisen rivin mahdollisesti aiheuttamia virheitä. On kuitenkin kohtuullista odottaa, että suurempi testikattavuus 
lisää todennäköisyyttä, että nämäkin tilanteet tulevat katettua.

    Eri ohjelmistot kehitetään lähtökohtaisesti eri tarkoituksiin, eri kehittäjien ja testaajien toimesta. 
Testikattavuus on tärkeä suhteellistaa näihin olosuhteisiin. Kontekstit ovat harvemmin samoja. Kokeneet kehittäjät 
kirjoittavat yleensä laadullisesti parempia testejä, koska kokemuksen karttuminen kasvattaa testikattavuutta ja näin 
ollen vähentää ohjelmiston virhetiheyttä. Inhimillisillä tekijöillä on siis valtava merkitys ohjelmiston laadullisissa 
tekijöissä.

    Mockus ym. toteavat testikattavuuden olevan käytännöllinen ja järkevä keino mitata ja varmistaa ohjelmiston laatua. 
Valitettavasti täydellisen testikattavuuden saavuttaminen ei ole todennäköisesti järkevää, sillä sen lopullinen 
hyödyllisyys näyttää olevan jopa negatiivinen ja samalla suurempien testikattavuuksien saavuttaminen 
haasteellista.\newline

\noindent Mockus ym. päätyvät seuraaviin neljään johtopäätökseen testikattavuudesta:
    
\begin{enumerate}

    \item Ohjelmiston hyväksymäkriteereihin tulisi kuulua 70 \% testikattavuus.
          
    \item Yli 70 \% testikattavuus ei yleensä ole tehokkuuden kannalta järkevää. Suuremman testikattavuuden 
          saavuttaminen kasvaa eksponentiaalisesti mitä suurempia testikattavuuksia tähdätään. Samalla 
          virheiden määrä näyttää laskevan vain lineaarisesti.
          
    \item Yli 70 \% testikattavuus vaatii haasteellista poikkeustenkäsittelyä lähde\-koodissa.
    
    \item Lopulta 70 \% testikattavuus on vain suuntaa antava luku, tehokkuuden kannalta optimaalisin testikattavuus 
          vaihtelee suuntaan tai toiseen ohjelmistosta riippuen.

\end{enumerate}

\subsubsection{Testikattavuuden pätevyyteen vaikuttavia tekijöitä}

Testikattavuuden pätevyyteen vaikuttaa samat tekijät, jotka vaikuttivat koodikirnun ja verkkoanalyysin pätevyyteen. 
Empiiristä tutkimuksista on vaikea tehdä yleisiä päätelmiä niiden kontekstisidonnaisen luonteen takia. Tutkimuksen 
pätevyyttä tukevoittaa kuitenkin se, että tapaustutkimuksessa tutkittiin kahta täysin erilaista ohjelmistoa. 
Ohjelmistojen takana oli eri organisaatio, sovellusala, ohjelmointikieli ja koko. Samalla myös kehittäjätiimien koko oli 
eri ja käyttäjäkunta myös. Voidaankin olettaa, että testikattavuus soveltuu hyvin myös muihin ohjelmistoihin.

\section{Kehittäjien käytänteet}

Ohjelmiston koodin takana on aina ihminen: kehittäjien käytänteillä ja ohjelmiston kehitysmalleilla on suuri laadullinen 
merkitys. Avainkysymykseksi nousee paikallistaa ne käytänteet, joilla on selvä yhteys ohjelmiston laatuun.

    Vanhojen raskaaseen ennakkosuunniteluun pohjautuvien menetelmien, kuten vesiputousmallin, rinnalla on noussut uusia 
ketterän kehityksen malleja. Sfetsos ja Stamelos suorittivat katselmuksen ketterän kehityksen empiirisistä 
tutkimuksista~\cite{SS10}. Katselmuksessa käytiin läpi 46 tutkimusta kahdeksasta eri tutkimustietokannasta. He 
havaitsivat katselmuksessaan, että ketterien kehitysmallien hyöty on merkittävä laadullinen tekijä.

    Ketterän kehityksen manifesti (agile manifesto) on muodostunut ketterän kehityksen tavoitteiden 
ympärille~\cite{BBB01}. Se painottaa yksilöitä ja yksilöiden vuorovaikutusta, toimivan ohjelmiston merkitystä, asiakkaan 
merkitystä kehitysprosessin kriittisenä osana ja muutoksiin sopeutuvaa kehitystä. Näiden periaatteiden takaamiseksi, 
ketterän kehityksen malleille on muodostunut useita käytänteitä. Näillä kehittäjät pystyvät hallinnoimaan ja 
varmistamaan kehitystyötä, varsinkin laadullisesta näkökulmasta. Ketterän kehitys huomioi nimenomaan asiakkaan 
kehitysprosessin tärkeänä osatekijänä.

    Asiakkaan tarpeet tulisi kartoittaa ja taata koko kehitysjakson aikana. Lopulta tuotettavan tuotteen tulisi tuoda 
jotain arvoa asiakkaalle. Ohjelmiston vaatimuksia on vaikea määrittää kattavasti heti alusta lähtien, siksi ketterässä 
mallissa painotetaan muutoksien hyväksymistä. Ohjelmiston kehitys saattaa olla hyvin pitkäaikainen prosessi, tilanteet 
ja käyttötarkoitukset muuttuvat prosessin aikana. Asiakas saattaa havaita ohjelmiston kannalta tärkeitä asioita hyvinkin 
myöhään kehityksessä.

\subsection{Ketterä kehitys}

Ketterän kehityksen idea on mahdollistaa muutokset kehitystyössä. Kehitystyötä tehdään iteratiivisesti pienissä 
palasissa ja samanaikaisesti painotetaan menetelmiä, jotka kasvattavat ohjelmiston hallintaa ja laatua. Laadun hallinta 
ja valvonta tulee jakaa koko kehityksen ajalle. Jokaisen iteraation eli palasen jälkeen tulisi asiakkaalle toimittavaa 
osa ohjelmiston toiminnallisuudesta. Tällöin asiakas voi havaita jo aikaisessa vaiheessa mahdolliset ongelmat 
tavoitteidensa ja toivomusten osalta ja pyrkiä selventämään niitä kehittäjille.

    Ketterässä kehityksessä laadun valvonta ei ole pelkästään yhden henkilön tehtävä. Jokainen kehitystyöhön osallistuva 
henkilö tekee sitä jatkuvasti omalta osaltaan. Henkilöille pitää luoda ilmapiiri ja ympäristö, jossa tämä on 
mahdollista. Kehitystiimin ja asiakkaan välinen luottamus on ensisijaisen tärkeää. Jatkuva tiedonvälitys ja keskustelu 
on olennainen osa tämän saavuttamista. Asiakas on jatkuvasti kehityksessä mukana arvioiden sovelluksen soveltavuutta 
tarkoituksiinsa. Samanaikaisesti hänen tulee myös varmistaa tuotteen toimivuus käyttäjien kannalta.

    Hyvien käytänteiden ja laadullisesti järkevien ratkaisujen seuraaminen parantaa ohjelmiston suunnittelua ja laatua. 
Mahdollisia ongelmia tulee katselmoida mahdollisimman usein ja tiimin toimintoja kehittää näiden ongelmien osalta. 
Ketterän kehityksen mallit eivät kuitenkaan aina kerro yksiselitteisesti miten kehitys pitäisi toteuttaa. Ne luovat 
kehyksen, jonka pohjalta kukin ohjelmistokehittäjä ja tiimi rakentaa omaan tarkoitukseen toimivan 
kokonaisuuden~\cite{Kn07}.

    Nykyään eniten käytössä olevat ketterän kehityksen mallit ovat Scrum ja XP~\cite{SS10}. Scrum tarjoaa ketterälle 
kehitykselle toimivaksi osoitetun kehyksen ja XP lukuisia ketterään kehitykseen soveltuvia ohjelmistokehityksen 
käytäntöjä. Scrum keskittyy lähinnä kehityksen hallinnolliseen puoleen: miten ohjelmistokehitys tulisi suunnitella, 
hallinnoida ja ajoittaa. Tästä syystä Scrum ja XP tukevat hyvin toisiaan~\cite{Kn07}.

    Scrumissa yksittäisiä iteraatioita kutsutaan pyrähdyksiksi, eli Sprinteiksi. Jokainen Sprintti muodostuu yhdessä 
asiakkaan kanssa valituista ja priorisoiduista ominaisuuksista tai parannuksista, joita kehittäjien tulisi sen aikana 
pyrkiä toteuttamaan. Itse toteutusta tukee useita XP:n esittämiä käytäntöjä, muun muassa testilähtöinen kehitys, 
pariohjelmointi ja suunnittelupeli. Testilähtöisessä kehityksessä ohjelmiston kehityksessä testit kirjoitetaan ennen 
niiden toteuttavaa toiminnallisuutta. Pariohjelmoinnissa harjaannutetaan kehittäjien taitoja ratkomalla ongelmia yhdessä 
työskentelyparin kanssa ja suunnittelupelissä yritetään arvioida kehitykseen kuluvaa aikaa ja samalla havaita ja pohtia 
mahdollisia ongelmia. XP:n käytännöt on käytännössä nykyään sulautunut osaksi Scrumia~\cite{Kn07}.

\subsection{Testilähtöinen kehitys}

Testilähtöinen kehitys (test-driven development, TDD) koostuu lähdekoodin kirjoittamisesta testilähtöisesti ja koodin 
jatkuvasta parantamisesta eli refaktoroinnista. Refaktoroinnissa tehdään pieniä muutoksia ohjelmiston koodiin 
muuttamatta sen ulkopuolista toiminnallisuutta. Testilähtöisessä kehityksessä testit kirjoitetaan ennen itse 
toiminnallisuuden ohjelmoimista. Tämän on tarkoitus saada kehittäjä suunnittelemaan ja miettimään uusia 
toiminnallisuuksia ja niiden ongelmia ennen itse logiikan toteuttamista. Jatkuvalla refaktoroinnilla pyritään 
rakentamaan toiminnallisuudet paremmiksi, kehittämällä jatkuvasti ohjelmiston lähdekoodista parempaa.

    Hyväksymätestien lisäksi, joita käytetään ohjelmiston vaatimusten määrit\-telemiseen, testilähtöinen kehitys ja 
refaktorointi parantavat yleisesti ohjelmiston laatua. Suurin osa kokeista ja tapaustutkimusta osoittivat Sfetsoksen ja 
Stameloksen katselmuksen mukaan laajoja laadullisia parannuksia ohjelmistojen ulkoiseen~\cite{SS10}. Julkaisun 
jälkeiset virheet vähenivät 5 \% -- 45 \%, joissain tapauksissa jopa 50 \% -- 90 \%. Tapaustutkimukset osoittivat 
suurempaa parannusta kuin kokeet. Kokeiden osalta heikompi parannus voidaan selittää valvotun ympäristön ja ajallisten 
rajoitusten seurauksena. Pančur ja Ciglarič mainitsevat heikon parannuksen osasyyksi testilähtöisen kehityksen 
vaikeuden~\cite{PC11}. Testien kirjoittaminen ennen itse toiminnallisuutta vaatii harjaantumista, jota jälkeen 
kirjoitettujen testien osalta ei vaadita. Siksi testilähtöisen kehityksen hyödyn vaikutukset todennäköisesti ilmentyvät 
vasta myöhemmin. Vain muutama koe ei havainnut testilähtöisellä kehityksellä olevan merkittävää vaikutusta ohjelmiston 
ulkoiseen laatuun.

    Sisäinen laatu kasvoi testilähtöisen kehityksen ansiosta merkittävissä määrin. Lähdekoodin uudelleenkäytettävyys ja 
vaivannäkö testaamista varten parani. Osa tutkimuksista osoitti lopullisen ohjelmiston kehitysajan kasvavan, mutta 
samalla osa tutkimuksista huomasi kehityksen kokonaiskustannusten vähenevän. Tuottavuuden kannalta Sfetsoksen ja 
Stameloksen katselmoivat tutkimukset osoittivat ristiriitaisia tuloksia: osassa tuottavuus kasvoi, osassa tuottavuudelle 
ei tapahtunut merkittäviä muutoksia, osassa tuottavuus jopa laski.

\subsection{Pariohjelmointi}

Pariohjelmointi on erittäin sosiaalinen ja yhteistyöhön perustuva toimintamalli. Se keskittyy kehittäjien yksilöllisiin 
taitoihin, kokemukseen, ominaispiirteisiin ja persoonallisuuteen. Pariohjelmoinnin tarkoitus on jatkuva suunnittelu ja 
koodikatselmus kahden kehittäjän kesken. Kehittäjät kirjoittavat yhdessä ohjelmiston toiminnallisuutta. Tämä vähentää 
virheiden määrää ja parantaa ohjelmiston suunnittelua ja laatua~\cite{SS10}.

    Sfetsos ja Stamelos havaitsivat pariohjelmoinnin olevan yksi merkittävim\-mistä laadullista tekijöistä käytännön 
näkökulmasta. Koodin suunnittelu ja laatu kasvoi 15 \% -- 65 \%. Monimutkaisiin ja vaativiin ongelmiin pariohjelmointi 
tuotti selvästi parempaa koodia kuin ohjelmointi yksin. Pariohjelmoinnin todettiin myös parantavan tiimityöskentelyn 
laatua, tiedon ja taitojen parempaa siirtymistä yksilöltä toiselle, tehokkaampia ja paremmin suunniteltuja algoritmeja, 
moraalin kasvua ja luottavaisempia kehittäjiä.

    Pariohjelmoinnin todettiin kuitenkin vaativan enemmän vaivannäköä kehittäjiltä ja näin ollen myös pariohjelmointi 
kasvatti kehitystyön kustannuksia. Tehokkuus väheni lievästi ja kehitystyön aikataulutus vaikeutui ja samalla 
kehitystiimeissä havaittiin persoonallisuus kitkoja. Tutkimukset havaitsivat tiettyjen taito, tieto ja kokemuspiirteiden 
sopivan paremmin pariohjelmointiin. Erityisesti persoonallisuuspiirteillä havaittiin suuri merkitys: avomieliset ja 
vastuulliset yksilöt sekä monipuoliset persoonallisuudet ja temperamentit soveltuvat pariohjelmointiin paremmin.

\section{Metriikat käytänteiden tukena}

Beck ym. puolustavat ketterän kehityksen manifestissa ketterän kehityksen itse organisoituvaa luonnetta~\cite{BBB01}. 
Kun kehittäjille annetaan tarpeeksi tukea, ottavat he itse vastuun myös ohjelmiston laadullisista puolista. Jatkuva 
hyvien käytänteiden ja suunnitteluperiaatteiden seuraaminen johtaa lopulta ohjelmiston laadun kasvuun~\cite{SS10}. 
Tiimien tulee arvioida näiden onnistumista tarpeeksi usein, jotta mahdollisiin ongelmiin voidaan puuttua ja tiimin 
käytänteitä hienosäätää.

    Suunnittelupeli ja Sprintin suunnittelu kattavat toiminnan, jota voitaisiin verrata suoraan laadullisten
määritelmien kehykseksi. Testilähtöinen kehitys, pariohjelmointi ja jatkuva integraatio taas paneutuvat laadullisen 
toteutuksen puoliin~\cite{SS10}.

    Moni XP:n käytännöistä yhdessä, kuten suunnittelupeli, pariohjelmointi ja testilähtöinen kehitys paransivat 
ohjelmiston laatua Sfetsoksen ja Stameloksen katselmuksen mukaan. Suunnittelupelin havaittiin parantavan kehityksen 
työmäärän ajallista estimointia. Käytänteiden seuraamisesta havaittiin refaktoroinnin ja tuottavuuden kasvu. XP 
-käytänteiden havaittiin toimivan paremmin nimenomaan pienissä kehittäjätiimeissä.

    Tutkimukset koodikirnusta, verkkoanalyysista ja testikattavuudesta ohjelmiston laadullisina metriikoina toivat 
esille inhimillisen tekijän laadun takaamisessa~\cite{NB05, ZN08, MNDT09}. Kehittäjän tulee siis aktiivisesti itse 
vaikuttaa ohjelmiston laatuun. Sfetsoksen ja Stameloksen katselmuksen pohjalta voidaan olettaa, että ketterä kehitys on 
oivallinen käytäntö ohjelmiston laadun varmistamisessa.

    Nagappan ja Ball toivat esille koodikirnun ja ohjelmiston riippuvuuksien käyttämisen laadullisina 
metriikoina~\cite{NB05, NB07}. Testien kirjoittaminen tulisi kanavoida näiden metriikoiden ilmaisemiin virheherkkiin 
komponentteihin~\cite{MNDT09}. Voidaan nähdä, että testilähtöinen kehitys tukisi näiden metriikoiden tavoitteita 
osuvasti. Nagappan ja Ball painottivat erikseen myös hyviä versionhallinnan käytäntöjä ~\cite{NB05}. Ohjelmistoon tehdyt 
muutokset tulisi rekisteröidä pienissä palasissa versionhallintaan mahdollisimman aikaisin ja usein. Tällä säästetään 
kallisarvoista kehitystyön historiaa ja mahdollistetaan ongelmatapauksissa paluu vanhoihin toimiviin koodiversioihin.

    Zimmermannin ja Nagappanin esittämällä verkkoanalyysillä~\cite{ZN08} voitaisiin taas ohjata sekä testauksen, että 
suunnittelun varoja sinne missä ne ovat tärkeimpiä~\cite{NB07, MNDT09}. Samalla vähennettäisiin inhimillisten, johtajien 
tai muiden kehitystiimin jäsenten tietotaitoon liittyviä riskejä ja parannettaisiin näin ohjelmiston laatua.

\section{Yhteenveto}

Kehitysvaiheessa olevan ohjelmiston laadun varmistaminen on hankalaa~\cite{NB05, NB07, ZN08, MNDT09}. Automaattisesti 
analysoitavat metriikat tarjoavat yhden keinon kohdentaa resursseja laadun takaamiseksi. Nagappan ja Ball esittävät 
suhteellisen koodikirnu-tekniikan järjestelmän virhetiheyden ennakoimiseen [NB05]. Koodikirnu (code churn) mittaa ja 
ilmaisee määrällisesti ohjelmiston komponentteihin kohdistuvia muutoksia tietyn ajanjakson aikana. Komponentit jotka 
muuttuvat paljon ovat tutkimuksen mukaan muita herkempiä virheille.

    Zimmermann ja Nagappan esittävät verkkoanalyysin suorittamista komponenttien riippuvuusverkoille~\cite{ZN08}. 
Verkkoanalyysillä voidaan paikallistaa ohjelmiston kriittiset komponentit. Keskeisessä roolissa olevat komponentit sekä 
yksittäiset komponentit, joilla on suuri määrä keskinäisiä riippuvuuksia, ovat yleisesti herkempiä virheille.

    Mockus, Nagappan ja Dinh-Trong tutkivat testien laadullista arviointia keinona havaita virheherkkiä komponentteja 
ohjelmistosta~\cite{MNDT09}. Testien analysoimisessa tulisi keskittyä nimenomaan niiden kykyyn havaita mahdollisia 
virheitä ohjelmistosta. Taustalla on olettamus, että jos jokin yksittäinen looginen ehto tai polku ei ole katettu 
vähintään yhdellä testillä, ei sen mahdollisesti sisältämiä virheitä pystytä havaitsemaan. Voidaankin olettaa, että 
suurempi testikattavuus löytää todennäköisesti enemmän virheitä ja takaa näin ollen paremman laadun.

    Ohjelmiston koodin takana on lopulta aina ihminen.  Laadun takeeksi ei voida luetella pelkästään mekaanisia laatua 
arvioivia metriikoita. Kehittäjän käytänteillä on suuri laadullinen merkitys ohjelman kaikissa kehitysvaiheissa, joten 
ohjelmistotuotantomenetelmät nousevat suureen rooliin. Niiden tulisi ohjata laadukasta kehitystä.

    Vanhojen raskaaseen ennakkosuunniteluun pohjautuvien tuotantomenetelmien, kuten vesiputousmallin, rinnalla on 
noussut uusia ketterän kehityksen malleja. Ne painottavat yksilöitä ja yksilöiden vuorovaikutusta, toimivan ohjelmiston 
merkitystä, asiakkaan merkitystä kehitysprosessin kriittisenä osana ja muutoksiin sopeutuvaa kehitystä~\cite{BBB01}. 
Useat empiiriset tutkimukset tukevat ketterien kehitysmallien hyötyä merkittävänä laadullisina tekijänä~\cite{SS10}. 

% --- Back matter ---

\bibliographystyle{babalpha}
\bibliography{../lahteet}

\end{document}