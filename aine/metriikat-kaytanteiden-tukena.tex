\documentclass[finnish]{../tktltiki2}

% --- Packages ---

\usepackage[utf8]{inputenc}
\usepackage{lmodern}
\usepackage{microtype}
\usepackage{amsfonts,amsmath,amssymb,amsthm,booktabs,color,enumitem,graphicx}
\usepackage[pdftex,hidelinks]{hyperref}

% Automatically set the PDF metadata fields
\makeatletter
\AtBeginDocument{\hypersetup{pdftitle = {\@title}, pdfauthor = {\@author}}}
\makeatother

% --- Language ---

\usepackage[fixlanguage]{babelbib}
\selectbiblanguage{finnish}

% Add bibliography to the table of contents
\usepackage[nottoc,numbib]{tocbibind}

% tocbibind renames the bibliography, use the following to change it back
\settocbibname{Lähteet}

% --- Theorem environment definitions ---

\newtheorem{lau}{Lause}
\newtheorem{lem}[lau]{Lemma}
\newtheorem{kor}[lau]{Korollaari}

\theoremstyle{definition}
\newtheorem{maar}[lau]{Määritelmä}
\newtheorem{ong}{Ongelma}
\newtheorem{alg}[lau]{Algoritmi}
\newtheorem{esim}[lau]{Esimerkki}

\theoremstyle{remark}
\newtheorem*{huom}{Huomautus}

% --- tktltiki2 options ---

\title{Metriikat käytänteiden tukena ohjelmiston laadun\\arvioimisessa}
\author{Kasper Hirvikoski}
\date{\today}
\level{Aine - Luonnosversio}

\begin{document}

% --- Front matter ---

\maketitle

\tableofcontents
\newpage

% --- Main matter ---

\section{Johdanto}

Ohjelmistot kehittyvät elinkaarensa aikana muun muassa uusien vaatimusten, optimisaatioiden, tietoturvaparannusten ja 
virhekorjausten johdosta. Kehitysvaiheessa olevan ohjelmiston laadun varmistaminen on hankalaa~\cite{NB05, ZN08, MND09}. 
Ohjelmiston testaamisen ja käytännössä havaittujen virheiden välillä on usein suuri kuilu. Virheiden määrää ei yleensä 
pystytä laskemaan luotettavasti ennen kuin tuote on valmis ja julkaistu asiakkaalle. Tässä piilee kuitenkin ongelman 
ydin: virheiden korjaaminen kehityksen lopussa on usein erittäin kallista.

    On selvää, että laadun varmistaminen ja mahdollisten ongelmakohtien havaitseminen mahdollisimman aikaisessa 
vaiheessa hyödyttää kehitystyötä suuresti. Ohjelmiston koodin takana on aina ihminen. Laadun takeeksi ei voida luetella 
pelkästään mekaanisia, laatua arvioivia metriikoita. Kehittäjän käytänteillä on suuri laadullinen merkitys ohjelman 
kehitysvaiheissa.

\section{Laadullinen arviointi}

Laadun varmistamista rajaa ohjelmistokehityksessä aika ja raha~\cite{ZN08}. Kehittäjät kohtaavat usein tiukkoja 
määräaikoja ja rajallisia henkilöresursseja laadun takaamiseen. Johtavat käyttävät omakohtaisia kokemuksiaan resurssien 
tehokkaaseen jakamiseen. Monimutkaisiin komponentteihin on syytä varata enemmän aikaa että rahaa. Komponenttien 
testausta ja tarkastusta tulee ohjata näihin osa-alueisiin. Johtajilla ei kuitenkaan ole läheskään aina tarvittavaa 
tietoa tai kokemusta, joiden pohjalta päätöksiä voitaisiin järkevästi tehdä. Päätökset tehdään usein odotusten mukaan ja 
näin ollen johtajat itse arvioivat laatua.

    Erilaiset metriikat tarjoavat yhden tehokkaan keinon ohjelmistojen laadun määrittelyyn. Staattisten ja dynaamisten 
virheenpaikannustekniikoiden johdosta virheiden laatu on muuttunut. Suurin osa virhetietokantoihin tallennetuista 
raporteista on nykyään luonteeltaan semanttisia, eli virheet koostuvat loogisista ongelmista~\cite{ZN08}. Metriikoiden 
tulee ottaa tämä huomioon.

    Nagappan ja Ball esittävät suhteellisen koodikirnu-tekniikan järjestelmän virhetiheyden ennakoimiseen~\cite{NB05}. 
Koodikirnu (code churn) mittaa ja ilmaisee määrällisesti ohjelmiston komponentteihin kohdistuvia muutoksia tietyn 
ajanjakson aikana. Nagappan ja Ball tuovat esille joukon suhteellisia koodikirnu-mittayksiköitä, jotka he rinnastavat 
muihin muuttujiin kuten komponenttien kokoon tai muokkauksen ajalliseen pituuteen. Käyttäen apuna tilastollisia 
regressiomalleja, he osoittavat suhteellisten koodikirnu-mittojen kyvyn havaita järjestelmän virhetiheyden paremmin kuin 
ehdottomien mittojen. Väittämien tueksi he suorittivat tapaustutkimuksen, jonka kohteena oli Windows Server 2003. 
Samalla he osoittavat, että relatiivinen koodikirnu pystyy paikallistamaan virheherkät komponentit 89 \% tarkkuudella.

    Monimutkaisuusmetriikat keskittyvät harvoin komponenttien välisiin vuorovaikutussuhteisiin. Ohjelmiston komponentit 
riippuvat lähes aina toisista komponenteista. Järjestelmän riippuvuudet voidaankin esittää matalan tason verkkoina, 
jossa komponenttien keskinäiset suhteet paljastuvat.

    Zimmermann ja Nagappan esittävät verkkoanalyysin suorittamista näille verkoille~\cite{ZN08}. Verkkoanalyysillä 
voidaan paikallistaa ohjelmiston kriittiset komponentit, jotka ovat muita virheherkempiä. Ohjelmistojen kohdalla 
komponentit muodostavat toimijat ja komponenttien väliset riippuvuudet toimijoiden vuorovaikutukset.

    Zimmerman ja Nagappan vertasivat verkkoanalyysia ja monimutkaisuusmetriikoita Windows Server 2003:n arvioimisessa. 
Verkkoanalyysillä saavutetaan heidän tutkimuksensa mukaan 10 \% parempi hyötyaste kuin pelkillä komponenttien 
monimutkaisuutta mittaavilla metriikoilla. Zimmermannin ja Nagappanin havaitsivat myös, että verkkometriikat pystyvät 
identifioimaan 60 \% komponenteista, joita ohjelmiston kehittäjät pitävät kriittisinä. Tämä on kaksi kertaa enemmän kuin 
tavallisilla monimutkaisuusmetriikoilla.

    Ohjelmiston kehityksessä koodin testaaminen on kriittinen osa laadun takaamista. Parhaassa tapauksessa testit 
löytävät virheet ohjelmistosta ennen kuin se julkaistaan asiakkaalle. Mockus, Nagappan ja Dinh-Trong tutkivat testien 
laadullista arviointia keinona havaita virheherkkiä komponentteja~\cite{MND09}. Testien analysoimisessa tulisi keskittyä 
nimenomaan testien kykyyn havaita mahdollisia virheitä ohjelmistosta. On selvää, että taitavat kehittäjät 
todennäköisesti tuottavat tehokkaampia testejä, mutta testien arvioiminen määrällisesti ja laadullisesti on kannattavaa 
automaattisesti. Yleisin testien tehokkuutta arvioiva mittari on testikattavuus.

    Testikattavuuden lajeja on useita. Yksinkertaisista luokka-, funktio-/metodi- ja käskykattavuuksista kehittyneisiin 
haara- ja polkukattavuuksiin. Taustalla on olettamus, että jos jokin yksittäinen ehto tai polku ei ole katettu vähintään 
yhdellä testillä, ei sen mahdollisesti sisältämiä virheitä pystytä havaitsemaan~\cite{MND09}. Tästä seuraa se, että 
suurempi testikattavuus löytää todennäköisesti enemmän virheitä ja takaa näin ollen parempaa laatua. Kattavuus kuvaa
yksinkertaisesti osuutta siitä kuinka monta riviä ohjelmakoodia on katettu sitä testaavalla testikoodilla. Se ei pysty 
arvioimaan kuinka todennäköisesti nämä rivit aiheuttavat virheen. Suuri testikattavuus ei siis yksinään takaa laatua. On 
kuitenkin selvää, että se auttaa laadun takaamisessa. Muita metriikoita tulisi käyttää kohdentamaan testejä kriittisiin 
osa-alueisiin.

    Ohjelmiston koodin takana on aina ihminen.  Laadun takeeksi ei voida luetella pelkästään mekaanisia laatua arvioivia 
metriikoita. Kehittäjän käytänteillä on suuri laadullinen merkitys ohjelman kehitysvaiheissa. 
Ohjelmistotuotantomenetelmät nousevat suureen rooliin. Niiden tulisi ohjata laadukasta kehitystä. Vanhojen raskaaseen 
ennakkosuunniteluun pohjautuvien menetelmien, kuten vesiputousmallin, rinnalla on noussut uusia ketterän kehityksen 
malleja. Ne painottavat yksilöitä ja yksilöiden vuorovaikutusta, toimivan ohjelmiston merkitystä, asiakkaan merkitystä 
kehitysprosessin kriittisenä osana ja muutoksiin sopeutuvaa kehitystä~\cite{BBB00}. Useat tutkimukset tukevat ketterien 
kehitysmallien hyötyä merkittävänä laadullisina tekijänä~\cite{SS10}. 

\section{Metriikat}

\subsection{Koodikirnu}

Nagappan ja Ball esittävät ohjelmistojen virhetiheyden arvioimiseen ratkaisuksi koodikirnua. Se mittaa ohjelmiston 
komponenttien ohjelmakoodiin kohdistuvien muutosten määrää tietyn ajanjakson aikana. Muutosten määrä on helposti 
ohjelmiston versiohallintajärjestelmien muutoshistoriasta. Useimmat versiohallintajärjestelmät vertailevat 
lähdekooditiedostojen historiaa ja laskevat automaattisesti koodiin kohdistuvia muutoksia. Nämä muutokset ilmentävät 
kuinka monta riviä tiedostoon on ohjelmoijan toimesta lisätty, poistettu tai muutettu sitten viimeiseen versiohistoriaan 
tallennetun version. Nämä muutokset muodostavat koodikirnun pohjan.

    Nagappan ja Ball esittävät joukon suhteellisia koodikirnu-mittoja virhetiheyden havaitsemiseen. Mitat ovat 
normalisoituja arvoja koodikirnun aikana saaduista tuloksista. Normalisoinnilla niistä on pyritty poistamaan mahdolliset 
häiriötekijät. Näitä mittoja on muun muassa yhteenlaskettujen koodirivien määrä, tiedostojen muutokset ja tiedostojen 
määrä. Tutkimukset ovat osoittaneet, että ehdottomat mittayksiköt, kuten pelkkä koodirivien summa, ovat huonoja ohjelmiston 
laadullisia ennusteita. Yleisesti ottaen ohjelmiston kehitysprosessia mittaavien yksiköiden on havaittu olevan parempia
osoittimia vikojen määrästä kuin pelkkää koodia arvioivat tekijät.

    Ohjelmistoa kehitettäessä sen komponenttien monimutkaisuus muuttuu. Monimutkaisuuden kasvun suhde on hyvä mittari 
virheherkkyyden kasvulle. Koodikirnu-mittojen on havaittu korreloivan selvästi ohjelmistoista tehtyjen vikailmoitusten 
kanssa. Mittojen välillä on havaittavissa myös keskinäisiä suhteita, joita voidaan mallintaa verkkoina. Yksinään 
kyseiset mittarit eivät välttämättä tuota toivottua tulosta. Näin ollen mittoja verrataan keskenään mahdollisten 
ristiriitaisuuksien havaitsemiseksi. On kuitenkin selvää, että johtopäätöksiin päätyminen on hankalaa empiirissä 
tutkimuksissa, koska prosessien taustalla on usein laajoja kontekstisidonnaisia tekijöitä.

\subsubsection{Ohjelmiston virheherkkyyteen vaikuttavat mitat}

Nagappan ja Ball listaavat seuraavat ehdottomat mitat koodikirnun pohjaksi. Nämä muodostavat suhteellisille mitoille 
vertailukohdat ohjelmiston virheherkkyyden analysoimisessa. Ehdottomat mitat eivät yksinään tuota luotettavaa tulosta.

\begin{description}
    
    \item[Yhteenlaskettu koodirivien määrä,] ohjelman uuden version ei-kommen\-toitujen koodirivien summa kaikkien 
    lähdekooditiedostojen kesken.
    
    \item[Käsiteltyjen koodirivien määrä,] ohjelman lähdekoodiin lisättyjen ja muuttuneiden koodirivien summa 
    edelliseen versioon nähden.
    
    \item[Poistettujen koodirivien määrä,] ohjelman lähdekoodista poistettujen koodirivien määrä edelliseen versioon 
    nähden.
    
    \item[Tiedostojen määrä,] yhden ohjelman kääntämiseen tarvittavien lähdekoodi\-tiedostojen määrä.
    
    \item[Muutosten ajanjakso,] yhteen tiedostoon kohdistuneiden muutosten ajanjakson pituus.
    
    \item[Muutosten määrä,] ohjelman tiedostoihin kohdistuneiden muutosten määrä edelliseen versioon nähden.
    
    \item[Käsiteltyjen tiedostojen määrä,] ohjelman käsiteltyjen tiedostojen yhteenlaskettu määrä.

\end{description}

    Näiden pohjalta he muodostivat kahdeksan suhteellista koodikirnu-mittaa ja osoittivat, että nämä mitat korreloivat 
selvästi kohonneeseen virhemäärään koodirivejä kohden. He käyttivät Spearmanin järjestyskorrelaatiokerrointa, joka kuvaa 
kahden asian keskinäistä vastaavuutta. Analyysissään he havaitsivat suhteellisten mittojen ylivertaisuuden ehdottomiin 
verrattuna. Empiiristen tutkimusten avulla he havaitsivat seuraavien mittojen soveltuvuuden todellisen virhetiheyden 
ennakoimiseen.

\begin{enumerate}
    
    \item {\bf Käsiteltyjen koodirivien määrä / Yhteenlaskettu koodirivien määrä}
    
    Suurempi osa käsiteltyjä koodirivejä suhteessa yhteenlaskettuun koodirivien määrän vaikuttaa yksittäisen ohjelman 
    virhetiheyteen.
    
    \item {\bf Poistettujen koodirivien määrä / Yhteenlaskettu koodirivien määrä}
    
    Suurempi osa poistettuja koodirivejä suhteessa yhteenlaskettuun koodirivien määrään vaikuttaa yksittäisen ohjelman 
    virhetiheyteen. Nagappan ja Ball havaitsivat korkean korrelaation mittojen 1. ja 2. välillä.
    
    \item {\bf Käsiteltyjen tiedostojen määrä / Tiedostojen määrä}
    
    Suurempi osa käsiteltyjä tiedostoja suhteessa ohjelman rakentavien tiedostojen lukumäärään lisää todennäköisyyttä, 
    että nämä käsitellyt tiedostot aiheuttavat uusia vikoja. Esimerkiksi meillä on kaksi ohjelmaa A ja B, jotka molemmat 
    koostuvat 20 lähdekooditiedostosta. A sisältää viisi käsiteltyä tiedostoa ja B kaksi. Todennäköisyys sille, että 
    muutokset ohjelmaan A saattavat aiheuttaa uusia vikoja on siis suurempi.
    
    \item {\bf Muutosten määrä / Käsiteltyjen tiedostojen määrä}
    
    Mitä suurempi määrä yksittäisiin tiedostoihin on kohdistunut muutoksia, sitä suurempi on todennäköisyys sille, että 
    tämä vaikuttaa kyseisistä lähdekooditiedostoista muodostuvan ohjelman virhetiheyteen. Esimerkiksi jos ohjelman A 
    viittä lähdekooditiedostoa on muutettu 20 kertaa ja ohjelman B viittä tiedostoja on muutettu kymmenen kertaa, 
    todennäköisyys sille, että ohjelma A sisältää uusia vikoja on suurempi.

    \item {\bf Muutosten ajanjakso / Tiedostojen määrä}
    
    Mitä pitempi aika on kulutettu muutoksiin, jotka kohdistuvat pieneen joukkoon tiedostoja, sitä suurempi on 
    todennäköisyys sillä, että nämä tiedostot sisältävät monimutkaisia rakenteita. Monimutkaisuus vaikuttaa koodin 
    helppoon ylläpidettävyyteen ja näin ollen lisää näiden tiedostojen aiheuttamaa virhetiheyttä.

    \item {\bf Käsiteltyjen ja poistettujen koodirivien määrä / Muutosten ajanjakso}
    
    Käsiteltyjen ja poistettujen koodirivien määrä suhteessa muutosten ajanjaksoon mittaa muutoksen määrää, jota pelkkä 
    muutosten ajanjakso ei yksinään ilmaise. Tätä mittaa tulee verrata mittaan 5. Oletuksena on, että mitä suurempi 
    määrä käsiteltyjä ja poistettuja koodirivejä on, sitä pitempi muutosten ajanjakson tulisi olla. Tämä taas vaikuttaa 
    ohjelman virhetiheyteen.

    \item {\bf Käsiteltyjen koodirivien määrä / Poistettujen koodirivien\\määrä}
    
    Ohjelmiston kehitys ei koostu pelkästään vikojen korjaamisesta vaan myös uuden kehittämisestä. Uusien ominaisuuksien 
    kehittämisessä käsiteltyjen koodirivien määrä on suhteessa suurempi kuin poistettujen koodirivien määrä. Suuri arvo 
    tälle mitalle ilmaisee uutta kehitystä. Saatua arvoa verrataan mittoihin 1. ja 2., jotka yksinään eivät ennakoi 
    uutta kehitystä.

    \item {\bf Käsiteltyjen ja poistettujen koodirivien määrä / Muutosten määrä}
    
    Mitä suurempi muutoksen laajuus on suhteessa muutosten määrään, sitä suurempi virhetiheys on. Mitta 8. toimii
    verrokkina mitoille 3.-6. Suhteessa mittoihin 3. ja 4., mitta 8. ilmaisee todellisen muutoksen määrää. Se kompensoi
    sitä tietoa, että yksittäisiä tiedostoja ei käsitellä toistuvasti pienten korjausten takia. Suhteessa mittoihin 5. 
    ja 6.,mitä suurempi käsiteltyjen ja poistettujen koodirivien määrä on käsittelyä kohden, sitä pitempi muutosten 
    ajanjakso tarvitaan ja sitä enemmän muutoksia kohdistuu esimerkiksi jokaista viikkoa kohden. Muussa tapauksessa 
    suuri määrä muutoksia on saattanut kohdistua hyvin lyhyeen ajanjaksoon, joka ennakoi suurempaa virhetiheyttä.

\end{enumerate}

\subsubsection{Johtopäätökset koodikirnusta}

Nagappan ja Ball havaitsivat, että koodi joka muuttuu useasti ennen julkaisua on selvästi virheherkempää kuin koodi, 
joka muuttuu vähemmän saman ajanjakson aikana. He tutkivat kahden ohjelmistojulkaisun Windows Server 2003 ja Windows 
Server 2003 Service Pack 1 pohjalta saatuja tuloksia. Julkaisuista analysoitiin 44,97 miljoonaa riviä koodia, joka 
muodostui 96 189 lähdekooditiedostosta. Niistä käännettiin 2 465 yksittäistä ohjelmaa.\newline

\noindent He päätyivät tutkimuksessaan neljään johtopäätökseen:

\begin{enumerate}

    \item Suhteellisten koodikirnu-mittojen nousua seuraa ohjelmiston virheherkkyyden kasvu.

    \item Suhteelliset mitat ovat parempia laadullisia arvioijia kuin ehdottomat mitat.

    \item Suhteellinen koodikirnu on tehokas tapa arvioida ohjelmiston virheherkkyyttä.

    \item Suhteellinen koodikirnu pystyy havaitsemaan virheherkän ja toimivan komponentin toisistaan.

\end{enumerate}

\subsubsection{Koodikirnun pätevyyteen vaikuttavia tekijöitä}

Nagappan ja Ball toteavat, että mittausvirheet vaikuttavat arvion luomiseen. Ongelma ei ole kuitenkaan suuri, sillä 
versiohallintajärjestelmät hoitavat automaattisesti analyysiin vaadittavat lähtöarvot. Koodikirnu vaatii kuitenkin myös 
ohjelmiston kehittäjältä hyviä käytäntöjä. Jos kehittäjä on tehnyt useita muutoksia rekisteröimättä niitä 
versiohallintajärjestelmän historiaan, osa muutoksista jää näkemättä. Kehittäjän toimista riippuen myös muutosten 
ajanjakson pituus voi selvästi pidentyä, jos muutoksia ei hyväksytä tarpeeksi aikaisin versiohallintajärjestelmään. 
Mittojen vertaaminen keskenään lieventää tästä johtuvia poikkeamia.

    He toteavat myös, että tapaustutkimuksen pätevyyteen voidaan nähdä vaikuttavan myös sen tosiasian, että 
tutkimuksessa analysoitiin vain yhtä ohjelmistojärjestelmää. Tämä ohjelmistojärjestelmä kuitenkin koostuu lukuisista 
komponenteista ja suuresta määrästä koodia. Analyysi on siis itsessään erittäin kattava.

\subsection{Verkkometriikat}

Verkkometriikat tarkemmin.

\subsection{Testikattavuus}

Testikattavuus tarkemmin.

\section{Kehittäjien käytänteet}

Ohjelmiston koodin takana on aina ihminen! Kehittäjien käytänteillä ja kehitysmalleilla on siis suuri merkitys.

\subsection{Ketterä kehitys}

Scrum, XP, jne.

\subsection{Versionhallinta}

"Commit early, commit often."

\subsection{Testilähtöinen kehitys}

Testilähtöisen kehityksen hyödyt ja haitat.

\subsection{Pariohjelmointi}

Pariohjelmoinnin hyödyt ja haitat.

\section{Metriikat käytänteiden tukena}

Rinnastetaan metriikat käytänteiden tueksi.

\section{Yhteenveto}

Ohjelmiston koodin takana on aina ihminen.  Selvästi siis laadun takeeksi ei voida luetella pelkästään mekaanisia laatua 
arvioivia metriikoita. Kehittäjän käytänteillä on suuri laadullinen merkitys ohjelman kehitysvaiheissa.

% --- Back matter ---

\bibliographystyle{babalpha}
\bibliography{../lahteet}

\end{document}